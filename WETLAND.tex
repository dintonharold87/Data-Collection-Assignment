\documentclass{article}
\begin{document}
\begin{titlepage}
	\begin{center}
	\line(1,0){300}\\
	[0.25in]
	\huge{\bfseries Pollution of Wetlands/Swamps in the different Regions of Uganda}\\
	[2mm]
	\line(1,0){200}\\
	\end{center}
\begin{flushright}
\textsc{\large Ainemukama Dinton Harold}
 
 COMPUTER SCIENCE YEAR $2$\\
$216007270$\\
$16/U/3020/PS$\\
Feb 23, 2018\\
\end{flushright}
	
\end{titlepage}
\tableofcontents
\thispagestyle{empty}
\cleardoublepage

\setcounter{page}{1}
\section{INTRODUCTION}
\subsection{Background to the problem}
Our country is comprised of so many wetlands/swamps in each of the different regions of the country.And all life in the country as we know depends on water to survive be it animals or human beings.Yet water pollution is a very real threat to our survival.It is considered the world's biggest health risk,threatening not only humans, but also the other plants and animals that rely on water to survive.
\subsection{Problem Statement}
Due to this problem,I decided to collect data on the different swamps in different regions of Uganda that is to say western region,central region,eastern region and northern region.This data includes the number of swamps in each region,the photograph of atleast one swamp in the region and where it is located on the map.My intention of collecting this kind of data was to find out how the different swamps are being polluted and which region in the country pollutes the swamps the most and how the problem of water pollution can be solved in Uganda.
\section{Objectives}
\subsection{General Objective}
To determine how the different swamps in different regions of Uganda are polluted by successive installation of sensors in drainage channels.
\subsection{Specific Objectives}
To determine the pollutants in the water.\\
To determine the significance of pollution of wetlands through drainage channels.\\
To determine the areas that pollute water most.\\
\section{Research Scope}
The target areas for this research concept paper, are the different swamps and wetlands in the four regions of Uganda namely:Western region, Eastern region, Central region and lastly Eastern region.
\section{Research Significance}
This research will be of great signicance to a point where we shall be able to identify the wetlands that are polluted the most. We shall also get to know how those areas are greatly affected by the pollution and also the habits that result into water pollution and it will as well be helpful to the government to control water pollution.
\section{Methodology}
Quite a number of methods will be used while carrying out this research especially when collecting data like the region, number of swamps/wetlands in the region, images of one of the swamps in the region and location. Interviews will come in handy especially when
asking for how people in the surrounding areas normally dispose off their rubbish.
\section{References}
MWE. (2011). Water and Environment Sector Performance Review
2011. Kampala: Ministry of Water and Environment (MWE).\\
MWE. (2013). National Water Resources Assessment 2013. Kampala:
Ministry of Water and Environment (MWE).\\
MWE. (2014). Water and Environment Sector Performance
Report 014. Kampala: Ministry of Water and Environment
(MWE).\\
NEMA (1998) State of Environment Report for Uganda. National Environment Management Authority (NEMA),
Kampala

\end{document}